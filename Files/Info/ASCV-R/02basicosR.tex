% Prof. Dr. Ausberto S. Castro Vera
% UENF - CCT - LCMAT - Curso de Ci\^{e}ncia da Computa\c{c}\~{a}o
% Campos, RJ,  2020
% Disciplina: Paradigmas de Linguagens de Programa\c{c}\~{a}o
% Aluno:


\chapter{ Conceitos b\'{a}sicos da Linguagem R}

Os livros b\'{a}sicos para o estudo da Linguagem R s\~{a}o: \cite{Cotton2013}, \cite{Kabacoff2015}, \cite{Wickham2016}, e \cite{Lander2014}

Neste cap\'{\i}tulo \'{e} apresentado ....

Segundo \cite{Sebesta2018}, a linguagem R,  . . .

De acordo com \cite{Sebesta2018} e \cite{roy04}, a linguagem R . . .

\cite{Sebesta2018} afirma que a linguagem R . . .

Considerando que a linguagem R (\cite{Sebesta2018}, \cite{wat90}) \'{e} considerada como ....

    %%%%%%%%=================================
    \section{Vari\'{a}veis e constantes}
    %%%%%%%%=================================


    %%%%%%%%=================================
    \section{Tipos de Dados B\'{a}sicos}
    %%%%%%%%=================================

     %%%%%%%%=================================
    \section{Tipos de Dados de Cole\c{c}\~{a}o}
    %%%%%%%%=================================

     %%%%%%%%=================================
    \section{Opera\c{c}\~{o}es L\'{o}gicas}
    %%%%%%%%=================================



    %%%%%%%%=================================
    \section{Estrutura de Controle e Fun\c{c}\~{o}es}
    %%%%%%%%=================================



    %%%%%%%%======================
    \section{M\'{o}dulos}
    %%%%%%%%======================



    %%%%%%%%======================
    \section{Orienta\c{c}\~{a}o a Objetos}
    %%%%%%%%======================


