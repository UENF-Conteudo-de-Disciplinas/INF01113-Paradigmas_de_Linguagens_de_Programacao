% Prof. Dr. Ausberto S. Castro Vera
% UENF - CCT - LCMAT - Curso de Ci\^{e}ncia da Computa\c{c}\~{a}o
% Campos, RJ,  2013-2019
% Disciplina: Paradigmas de Linguagens de Programa\c{c}\~{a}o

% Arquivo: arquivo01.tex



\section{Primeira se\c{c}\~{a}o do documento}
Este \'{e} um primeiro documento escrito utilizando o processador-linguagem \LaTeX{}. Para separar um par\'{a}grafo de outro \'{e} necess\'{a}rio deixar uma linha em branco.

Um bom livro texto de consulta sobre o  \LaTeX \'{e} \cite{Kottwitz2011}. Tamb\'{e}m pode ser consultado \cite{Mittelbach2004}

    \subsection{Comandos b\'{a}sicos}
    Todo documento em  \LaTeX devem ter os seguintes comandos b\'{a}sicos:
    \begin{verbatim}
    \documentclass[ops]{tipoDoc}
    ..................
    \begin{document}
    .................. texto principal ...

    \end{document}
    \end{verbatim}



\section{Listas de itens}
 \LaTeX permite criar tr\^{e}s tipos de listas: enumeradas, n\~{a}o enumeradas e descri\c{c}\~{o}es de itens
    \subsection{Listas enumerdas}

        \begin{enumerate}
        	\item Computadores
        	\item Impressoras
        	\item Redes
        \end{enumerate}



     \subsection{Listas n\~{a}o enumeradas}
       \begin{itemize}
       	\item Brasil
       	\item Bolivia
       	\item B\'{e}lgica
       	\item Bulgaria
       	\item Bosnia
       \end{itemize}



    \subsection{Listas de descri\c{c}\~{a}o}

       \begin{description}
       	\item[Redes com cabeamento] Usando cabos
       	\item[Redes Wireless] Redes sem fio usando protocolos 802.11abgn
       	\item[Redes Internet] Rede mundial de computadores
       \end{description}











\section{Tabular}
% % % % %---------------
Material tabular \'{e} o texto que inclui linhas e colunas: \\
\begin{tabular}{llc}
	fruta & cor  & Pre\c{c}o R\$ \\
	\hline
	lim\~{a}o & verde &  3,40 \\
	banana& amarelo & 1,85 \\
	mam\~{a}o & laranja & 5,20
\end{tabular}\\
l=left(alinhado a esquerda), r = right (alinhado a direita) c= center (centralizado)




\section{Tabelas}
% % % % %---------------
Observe que uma tabela, por exemplo, Tab.\ref{tabela} e \ref{tabela2},,   tem outro formato:

\begin{table}[H]     % H=aqui, nesta posi\c{c}\~{a}o dentro do texto
	\centering
	\caption{Exemplo de Frutas e seus pre\c{c}os} 	\label{tabela}
	\begin{tabular}{llc}
		fruta & cor  & Pre\c{c}o R\$ \\        %   o simbolo & separa colunas
		\hline
		lim\~{a}o & verde &  3,40 \\
		banana& amarelo & 1,85 \\
		mam\~{a}o & laranja & 5,20
	\end{tabular}\\
    {\small Fonte: O autor}
\end{table}

\begin{table}[H]     % H=aqui, nesta posi\c{c}\~{a}o dentro do texto
	\centering
	\caption{Exemplo de Frutas e seus pre\c{c}os} 	\label{tabela2}
	\begin{tabular}{|l|l|c|}
        \hline
		fruta & cor  & Pre\c{c}o R\$ \\        %   o simbolo & separa colunas
		\hline
		lim\~{a}o & verde &  3,40 \\
		banana& amarelo & 1,85 \\
		mam\~{a}o & laranja & 5,20\\
     	\hline
	\end{tabular}\\
    {\small Fonte: O autor}
\end{table}



\section{Figuras}
% % % % %---------------
As Fig.\ref{foto1}  e Fig.\ref{foto2} s\~{a}o exemplos de como inserir fotos e imagens em um documento \LaTeX.
\begin{figure}[H]
	\begin{center}
		\caption{Este \'{e} um exemplo de figura em \LaTeX}
		\label{foto1}
		\includegraphics[angle=90,width=0.3\textwidth]{latex.png}
		\includegraphics[angle=45,width=0.3\textwidth]{latex.png}	
		\includegraphics[width=4cm]{latex.png}	\\		
        {\small Fonte: O autor}
	\end{center}
\end{figure} 