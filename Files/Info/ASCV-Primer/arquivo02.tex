% Prof. Dr. Ausberto S. Castro Vera
% UENF - CCT - LCMAT - Curso de Ci\^{e}ncia da Computa\c{c}\~{a}o
% Campos, RJ,  2013-2019
% Disciplina: Paradigmas de Linguagens de Programa\c{c}\~{a}o

% Arquivo: arquivo02.tex




\section{Matem\'{a}ticas}
% % % % %---------------
Qualquer f\'{o}rmula matem\'{a}tica dentro de um par\'{a}grafo come\c{c}a e termina com o s\'{\i}mbolo \$, por exemplo, $ x^2 + y^2 = z^2$. Para uma f\'{o}rmula matem\'{a}tica, como por exemplo, uma equa\c{c}\~{a}o, tem que ser utilizados dois s\'{\i}mbolos \$\$ no in\'{\i}cio e final:
$$
  w = \sum_{i=1}^{n} (x_{i}+y_{i})^{2}
$$
ou utilizar o ambiente \textit{equation}  :
\begin{equation}
  W = \sum_{i=1}^{n} (A_{i}+B_{i})^{2}
\end{equation}
Observe-se que na primeira, temos apenas uma f\'{o}rmula matem\'{a}tica e na segunda, temos uma equa\c{c}\~{a}o (numerada). Para ter uma equa\c{c}\~{a}o n\~{a}o numerada temos que adicionar um asterisco e utilizar o pacote {\tt amsmath} no pre\^{a}mbulo:
\begin{equation*}
  \lim_{x \to 0} (x^{2} - 3x + 6) = 4
\end{equation*}

	\subsection{Matrizes}
	$$
	A = \begin{matrix}
        	\alpha     & \beta^{*}\\
	        \gamma^{*} & \delta
	    \end{matrix}
	$$

	$$
	A = \begin{bmatrix}
	\alpha     & \beta^{*}\\
	\gamma^{*} & \delta
	\end{bmatrix}
	$$

	$$
	A = \begin{Bmatrix}
	\alpha     & \beta^{*}\\
	\gamma^{*} & \delta
	\end{Bmatrix}
	$$

	$$
	A = \begin{pmatrix}
	\alpha     & \beta^{*}\\
	\gamma^{*} & \delta
	\end{pmatrix}
	$$

	$$
	A = \begin{vmatrix}
	\alpha     & \beta^{*}\\
	\gamma^{*} & \delta
	\end{vmatrix}
	$$

	$$
	A = \begin{Vmatrix}
	\alpha     & \beta^{*}\\
	\gamma^{*} & \delta
	\end{Vmatrix}
	$$

	$$
	A = \begin{bmatrix}
	1     & 2 & 3 & x & y\\
	4     & 5 & 6 & a & b\\
	7     & 8 & 9 & m & n
	\end{bmatrix}
	$$


\section{Usando refer\^{e}ncias bibliogr\'{a}ficas }
Existem muitos pacotes para gerenciar as refer\^{e}ncias bibliogr\'{a}ficas no \LaTeX, por\'{e}m, os 3 mais conhecidas s\~{a}o: bibtex, natbib e biblatex. Aqui mostraremos um exemplo de como usar o Bib\TeX.

Os livros textos de consulta mais utilizados s\~{a}o: \cite{Mittelbach2004}, \cite{Gratzer2014}, \cite{Kottwitz2011} e \cite{Dongen2012}.


Textos podem ser citados de diferentes maneiras, por exemplo:
\begin{itemize}
	\item \citep{Mittelbach2004}
	\item \citet{Mittelbach2004}
	\item \citeauthor{Mittelbach2004}
	\item \citealt{Mittelbach2004}
\end{itemize}
Para isto deve ser utilizado estes dois comandos:
\begin{verbatim}
\usepackage{natbib}
..............
\bibliographystyle{plainnat}
\end{verbatim}









\section{Ferramentas m\'{\i}nimas necess\'{a}rias}
Consideramos como ferramentas m\'{\i}nimas para trabalhar com \LaTeX as seguintes:
\begin{itemize}
	\item Compilador MikTeX e pacotes \LaTeX, v2.9.6361 \\
	\url{https://miktex.org/download#collapse264}

	\item Editor dirigido pela sintaxe para \LaTeX: TeXstudio v2.12.6\\
	\url{http://www.texstudio.org/}
	
	\item Gerenciador de Refer\^{e}ncias Bibliogr\'{a}ficas  JabRef v3.8.2 \\
	\url{https://www.fosshub.com/JabRef.html}
	
	\item Leitor de arquivos PDF, Adobe Reader\\
\end{itemize}


     \subsection{Links interessantes}
     \begin{itemize}
     	\item CTAN \\
     	\url{http://www.ctan.org/}
     	
     	\item wikiLaTeX \\
     	 \url{http://en.wikibooks.org/wiki/LaTeX}
     	
     	\item \LaTeX/Bibliography Management \\
           \url{http://en.wikibooks.org/wiki/LaTeX/Bibliography\_Management}
     	
     	\item Bib\TeX \\
     	 \url{http://en.wikipedia.org/wiki/BibTeX}
     	
     	\item ShareLaTeX \\
     	\url{https://pt.sharelatex.com/learn/Main\_Page}
     	
     	\item LaTeX wiki \\
     	\url{http://latex.wikia.com/wiki/Main\_Page}

        \item LaTeX Color Definitions \\
        \url{http://latexcolor.com/}
     	
     \end{itemize}



\begin{figure}[H]
	\begin{center}
		\includegraphics[width=0.3\textwidth]{latex2.png}
		\includegraphics[angle=45,width=0.3\textwidth]{latex3.png}		
		\caption{Este \'{e} outro exemplo de figura em \LaTeX}
		\label{foto2}
	\end{center}
\end{figure}


