% Prof. Dr. Ausberto S. Castro Vera
% UENF - CCT - LCMAT - Curso de Ci\^{e}ncia da Computa\c{c}\~{a}o
% Campos, RJ,  2013-2020
% Disciplina: Paradigmas de Linguagens de Programa\c{c}\~{a}o


\documentclass[12pt]{article}

% inicio do preambulo
% lista de pacotes a serem utilizados no documento
\usepackage[brazil]{babel}    % nomes e titulos em portugu\^{e}s
\usepackage[utf8]{inputenc}   % para os acentos direto do teclado
\usepackage{amsmath}          % para simbolos matem\'{a}ticos
\usepackage{here}
\usepackage{graphicx}         % para graficos e imagens

% % Prof. Dr. Ausberto S. Castro Vera
% UENF - CCT - LCMAT - Curso de Ci\^{e}ncia da Computa\c{c}\~{a}o
% Campos, RJ,  2013-2019
% Disciplina: Paradigmas de Linguagens de Programa\c{c}\~{a}o

% Arquivo: definicoes.tex


%

\usepackage{natbib}
\usepackage{hyperref}
\usepackage{xcolor}
\usepackage{tcolorbox}
\usepackage{url}

\usepackage[brazilian,hyperpageref]{backref}
% Configura\c{c}\~{o}es do pacote backref
% Usado sem a op\c{c}\~{a}o hyperpageref de backref
\renewcommand{\backrefpagesname}{Citado na(s) p\'{a}gina(s):~}
% Texto padr\~{a}o antes do n\'{u}mero das p\'{a}ginas
\renewcommand{\backref}{}
% Define os textos da cita\c{c}\~{a}o
\renewcommand*{\backrefalt}[4]{
	\ifcase #1 %
		Nenhuma cita\c{c}\~{a}o no texto.%
	\or
		Citado na p\'{a}gina #2.%
	\else
		Citado #1 vezes nas p\'{a}ginas #2.%
	\fi}%
% ---

\hypersetup{
    bookmarks=true,         % show bookmarks bar?
    unicode=false,          % non-Latin characters in Acrobat’s bookmarks
    pdftoolbar=true,        % show Acrobat’s toolbar?
    pdfmenubar=true,        % show Acrobat’s menu?
    pdffitwindow=false,     % window fit to page when opened
    pdfstartview={FitH},    % fits the width of the page to the window
    pdftitle={Modelo de Documento Simples \LaTeX},    % title
    pdfauthor={Ausberto S. Castro Vera - CC-LCMAT-CCT-UENF},     % author
    pdfsubject={Paradigmas de Linguagens de Programa\c{c}\~{a}o},   % subject of the document
    pdfcreator={ASCV, WinEdt},   % creator of the document
    pdfproducer={ASCV}, % producer of the document
    pdfkeywords={Linguagens} {LaTeX} {Documentos}, % list of keywords
    pdfnewwindow=true,      % links in new window
    colorlinks=true,       % false: boxed links; true: colored links
    linkcolor=red,          % color of internal links (change box color with linkbordercolor)
    citecolor=blue,        % color of links to bibliography
    filecolor=magenta,      % color of file links
    urlcolor=cyan           % color of external links
}	



\textwidth=15cm               % largura do texto


\usepackage{natbib}
\usepackage{hyperref}
\usepackage{xcolor}
\usepackage{tcolorbox}
\usepackage{url}

\usepackage[brazilian,hyperpageref]{backref}
% Configura\c{c}\~{o}es do pacote backref
% Usado sem a op\c{c}\~{a}o hyperpageref de backref
\renewcommand{\backrefpagesname}{Citado na(s) p\'{a}gina(s):~}
% Texto padr\~{a}o antes do n\'{u}mero das p\'{a}ginas
\renewcommand{\backref}{}
% Define os textos da cita\c{c}\~{a}o
\renewcommand*{\backrefalt}[4]{
	\ifcase #1 %
		Nenhuma cita\c{c}\~{a}o no texto.%
	\or
		Citado na p\'{a}gina #2.%
	\else
		Citado #1 vezes nas p\'{a}ginas #2.%
	\fi}%
% ---

\hypersetup{
    bookmarks=true,         % show bookmarks bar?
    unicode=false,          % non-Latin characters in Acrobat’s bookmarks
    pdftoolbar=true,        % show Acrobat’s toolbar?
    pdfmenubar=true,        % show Acrobat’s menu?
    pdffitwindow=false,     % window fit to page when opened
    pdfstartview={FitH},    % fits the width of the page to the window
    pdftitle={Modelo de Documento Simples \LaTeX},    % title
    pdfauthor={Ausberto S. Castro Vera - CC-LCMAT-CCT-UENF},     % author
    pdfsubject={Paradigmas de Linguagens de Programa\c{c}\~{a}o},   % subject of the document
    pdfcreator={ASCV, WinEdt},   % creator of the document
    pdfproducer={ASCV}, % producer of the document
    pdfkeywords={Linguagens} {LaTeX} {Documentos}, % list of keywords
    pdfnewwindow=true,      % links in new window
    colorlinks=true,       % false: boxed links; true: colored links
    linkcolor=red,          % color of internal links (change box color with linkbordercolor)
    citecolor=blue,        % color of links to bibliography
    filecolor=magenta,      % color of file links
    urlcolor=cyan           % color of external links
}	



\textwidth=15cm               % largura do texto

%opening
\title{\bf Modelo de Documento Simples \LaTeX}
\author{\textsf{Prof. Ausberto S. Castro Vera} \\
	            UENF - CCT - LCMAT \\
	            Ci\^{e}ncia da Computa\c{c}\~{a}o
	   }
\date{Campos, RJ, \today}    % data. Pode ser \date{Janeiro, 2015}

\begin{document}
\maketitle     %faz o titulo usando dados do preambulo


\begin{abstract}
Este documento do tipo ARTIGO apresenta as constru\c{c}\~{o}es b\'{a}sicas do \LaTeX{}. Aqui s\~{a}o apresentadas as constru\c{c}\~{o}es b\'{a}sicas deste processador de textos: comandos b\'{a}sicos, material tabular, tabelas, figuras, f\'{o}rmulas matem\'{a}ticas, etc. O editor utilizado \'{e} o TeXstudio v2.12.14
\end{abstract}

\newpage
\tableofcontents
\newpage




\section{Primeira se\c{c}\~{a}o do documento}
Este \'{e} um primeiro documento escrito utilizando o processador-linguagem \LaTeX{}. Para separar um par\'{a}grafo de outro \'{e} necess\'{a}rio deixar uma linha em branco.

Um bom livro texto de consulta sobre o  \LaTeX \'{e} \cite{Kottwitz2011}. Tamb\'{e}m pode ser consultado \cite{Mittelbach2004}

    \subsection{Comandos b\'{a}sicos}
    Todo documento em  \LaTeX devem ter os seguintes comandos b\'{a}sicos:
    \begin{verbatim}
    \documentclass[ops]{tipoDoc}
    ..................
    \begin{document}
    .................. texto principal ...

    \end{document}
    \end{verbatim}



\section{Listas de itens}
 \LaTeX permite criar tr\^{e}s tipos de listas: enumeradas, n\~{a}o enumeradas e descri\c{c}\~{o}es de itens
    \subsection{Listas enumeradas}

        \begin{enumerate}
        	\item Computadores
        	\item Impressoras
        	\item Redes
        \end{enumerate}



     \subsection{Listas n\~{a}o enumeradas}
       \begin{itemize}
       	\item Brasil
       	\item Bolivia
       	\item B\'{e}lgica
       	\item Bulgaria
       	\item Bosnia
       \end{itemize}



    \subsection{Listas de descri\c{c}\~{a}o}

       \begin{description}
       	\item[Redes com cabeamento] Usando cabos
       	\item[Redes Wireless] Redes sem fio usando protocolos 802.11abgn
       	\item[Redes Internet] Rede mundial de computadores
       \end{description}











\section{Tabular}
% % % % %---------------
Material tabular \'{e} o texto que inclui linhas e colunas: \\
\begin{tabular}{llc}
	fruta & cor  & Pre\c{c}o R\$ \\
	\hline
	lim\~{a}o & verde &  3,40 \\
	banana& amarelo & 1,85 \\
	mam\~{a}o & laranja & 5,20
\end{tabular}\\
l=left(alinhado a esquerda), r = right (alinhado a direita) c= center (centralizado)




\section{Tabelas}
% % % % %---------------
Observe que uma tabela, por exemplo, Tab.\ref{tabela} e \ref{tabela2},   tem outro formato:

\begin{table}[H]     % H=aqui, nesta posi\c{c}\~{a}o dentro do texto
	\centering
	\caption{Exemplo de Frutas e seus pre\c{c}os} 	\label{tabela}
	\begin{tabular}{llc}
		fruta & cor  & Pre\c{c}o R\$ \\        %   o simbolo & separa colunas
		\hline
		lim\~{a}o & verde &  3,40 \\
		banana& amarelo & 1,85 \\
		mam\~{a}o & laranja & 5,20
	\end{tabular}\\
    {\small Fonte: O autor}
\end{table}

\begin{table}[H]     % H=aqui, nesta posi\c{c}\~{a}o dentro do texto
	\centering
	\caption{Exemplo de Frutas e seus pre\c{c}os} 	\label{tabela2}
	\begin{tabular}{|l|l|c|}
        \hline
		fruta & cor  & Pre\c{c}o R\$ \\        %   o simbolo & separa colunas
		\hline
		lim\~{a}o & verde &  3,40 \\
		banana& amarelo & 1,85 \\
		mam\~{a}o & laranja & 5,20\\
     	\hline
	\end{tabular}\\
    {\small Fonte: O autor}
\end{table}

\section{Figuras}
% % % % %---------------
As Fig.\ref{foto1}  e Fig.\ref{foto2} s\~{a}o exemplos de como inserir fotos e imagens em um documento \LaTeX.
%%%%%%%%%%%%%%%%%%%%%%%%%%%%%%%%%%%%%%%
\begin{figure}[H]
	\begin{center}
		\caption{Este \'{e} um exemplo de figura em \LaTeX}
		\label{foto1}
		\includegraphics[angle=90,width=0.3\textwidth]{latex.png}
		\includegraphics[angle=45,width=0.3\textwidth]{latex.png}	
		\includegraphics[width=4cm]{latex.png}	\\		
        {\small Fonte: O autor}
	\end{center}
\end{figure}
%%%%%%%%%%%%%%%%%%%%%%%%%%%%%%%%%%%%%%%
\section{Matem\'{a}ticas}
% % % % %---------------
Qualquer f\'{o}rmula matem\'{a}tica dentro de um par\'{a}grafo come\c{c}a e termina com o s\'{\i}mbolo \$, por exemplo, $ x^2 + y^2 = z^2$. Para uma f\'{o}rmula matem\'{a}tica, como por exemplo, uma equa\c{c}\~{a}o, tem que ser utilizados dois s\'{\i}mbolos \$\$ no in\'{\i}cio e final:
$$
  w = \sum_{i=1}^{n} (x_{i}+y_{i})^{2}
$$
ou utilizar o ambiente \textit{equation}  :
\begin{equation}
  W = \sum_{i=1}^{n} (A_{i}+B_{i})^{2}
\end{equation}
Observe-se que na primeira, temos apenas uma f\'{o}rmula matem\'{a}tica e na segunda, temos uma equa\c{c}\~{a}o (numerada). Para ter uma equa\c{c}\~{a}o n\~{a}o numerada temos que adicionar um asterisco e utilizar o pacote {\tt amsmath} no pre\^{a}mbulo:
\begin{equation*}
  \lim_{x \to 0} (x^{2} - 3x + 6) = 4
\end{equation*}

	\subsection{Matrizes}
	$$
	A = \begin{matrix}
        	\alpha     & \beta^{*}\\
	        \gamma^{*} & \delta
	    \end{matrix}
	$$

	$$
	A = \begin{bmatrix}
	\alpha     & \beta^{*}\\
	\gamma^{*} & \delta
	\end{bmatrix}
	$$

	$$
	A = \begin{Bmatrix}
	\alpha     & \beta^{*}\\
	\gamma^{*} & \delta
	\end{Bmatrix}
	$$

	$$
	A = \begin{pmatrix}
	\alpha     & \beta^{*}\\
	\gamma^{*} & \delta
	\end{pmatrix}
	$$

	$$
	A = \begin{vmatrix}
	\alpha     & \beta^{*}\\
	\gamma^{*} & \delta
	\end{vmatrix}
	$$

	$$
	A = \begin{Vmatrix}
	\alpha     & \beta^{*}\\
	\gamma^{*} & \delta
	\end{Vmatrix}
	$$

	$$
	A = \begin{bmatrix}
	1     & 2 & 3 & x & y\\
	4     & 5 & 6 & a & b\\
	7     & 8 & 9 & m & n
	\end{bmatrix}
	$$


\section{Usando refer\^{e}ncias bibliogr\'{a}ficas }
Existem muitos pacotes para gerenciar as refer\^{e}ncias bibliogr\'{a}ficas no \LaTeX, por\'{e}m, os 3 mais conhecidas s\~{a}o: bibtex, natbib e biblatex. Aqui mostraremos um exemplo de como usar o Bib\TeX.

Os livros textos de consulta mais utilizados s\~{a}o: \cite{Mittelbach2004}, \cite{Gratzer2014}, \cite{Kottwitz2011} e \cite{Dongen2012}.


Textos podem ser citados de diferentes maneiras, por exemplo:
\begin{itemize}
	\item \citep{Mittelbach2004}
	\item \citet{Mittelbach2004}
	\item \citeauthor{Mittelbach2004}
	\item \citealt{Mittelbach2004}
\end{itemize}
Para isto deve ser utilizado estes dois comandos:
\begin{verbatim}
\usepackage{natbib}
..............
\bibliographystyle{plainnat}
\end{verbatim}









\section{Ferramentas m\'{\i}nimas necess\'{a}rias}
Consideramos como ferramentas m\'{\i}nimas para trabalhar com \LaTeX as seguintes:
\begin{itemize}
	\item Compilador MikTeX e pacotes \LaTeX, v2.9.6361 \\
	\url{https://miktex.org/download#collapse264}

	\item Editor dirigido pela sintaxe para \LaTeX: TeXstudio v2.12.6\\
	\url{http://www.texstudio.org/}
	
	\item Gerenciador de Refer\^{e}ncias Bibliogr\'{a}ficas  JabRef v3.8.2 \\
	\url{https://www.fosshub.com/JabRef.html}
	
	\item Leitor de arquivos PDF, Adobe Reader\\
\end{itemize}


     \subsection{Links interessantes}
     \begin{itemize}
     	\item CTAN \\
     	\url{http://www.ctan.org/}
     	
     	\item wikiLaTeX \\
     	 \url{http://en.wikibooks.org/wiki/LaTeX}
     	
     	\item \LaTeX/Bibliography Management \\
           \url{http://en.wikibooks.org/wiki/LaTeX/Bibliography\_Management}
     	
     	\item Bib\TeX \\
     	 \url{http://en.wikipedia.org/wiki/BibTeX}
     	
     	\item ShareLaTeX \\
     	\url{https://pt.sharelatex.com/learn/Main\_Page}
     	
     	\item LaTeX wiki \\
     	\url{http://latex.wikia.com/wiki/Main\_Page}

        \item LaTeX Color Definitions \\
        \url{http://latexcolor.com/}
     	
     \end{itemize}



\begin{figure}[H]
	\begin{center}
		\caption{Este \'{e} outro exemplo de figura em \LaTeX}
		\label{foto2}
		\includegraphics[width=0.3\textwidth]{latex2.png}
		\includegraphics[angle=45,width=0.3\textwidth]{latex3.png}	\\	
        {\small Fonte: O autor}
	\end{center}
\end{figure}



% % % % Bibliografia

\bibliographystyle{plainnat}
\bibliography{bibliogr}

\end{document}
