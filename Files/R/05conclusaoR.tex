\chapterimage{Conclusao.jpg} % Chapter heading image
\chapter{Conclusões}
  \begin{comment}
    Prof. Dr. Ausberto S. Castro Vera
    UENF - CCT - LCMAT - Curso de Ciência da Computação
    Campos, RJ,  2021
    Disciplina: Paradigmas de Linguagens de Programação
    Os problemas enfrentados neste trabalho ...
    O trabalho que foi desenvolvido em forma resumida ...
    Aspectos não considerados que poderiam ser estudados ou úteis para ...
    
    
    \begin{figure}[H]
    	\begin{center}
    		\caption{Aplicação da Linguagem R} \label{ling2}
    		\includegraphics[width=12cm]{R02.png} \\
    		{\tiny \sf Fonte: O autor }
    	\end{center}
    \end{figure}
  \end{comment}
  
  Como desfecho deste trabalho, vejo de forma positiva o ganho de conhecimento sobre a linguagem R e para você que leu e também a mim que escrevi. A mim foi ainda mais vantajoso pois obtive também conhecimento sobre o desenvolvimento de arquivos utilizando o \LaTeX. Percebi que durante a escrita, é complicado se ater aos "meios legado" de aquisição de conhecimento. Parte dessa dificuldade se dá por causa da facilidade e rapidez de se encontrar praticamente qualquer informação através de uma rápida pesquisa, ao invés de passar alguns minutos extras buscando PDFs online e vasculhando por todo um livro para descobrir se ele será útil para explicar uma questão específica ou não. Quanto a esta questão específica é outro ponto relevante, visto que as informações têm sido tão facilmente divulgadas nos últimos tempos, e o diálogo entre pessoas em locais distantes do globo se torno algo tão trivial, que uma boa parcela dos problemas mais comuns encontrados, principalmente na área da computação, já está bem documentada. Esses fatores acabam contribuindo para perda de parte do valor da pesquisa literária.
  
  Alguns aspectos que poderiam ter sido citados neste documento são: a visão dos programadores quanto a linguagem R em comparação a outras linguagens de propósito similar; a faixa salarial média de programadores na linguagem R por área; a explicação de como se obter os datasets apresentados no capítulo de aplicações; Uma progressão de aplicações indo do básico ao avançado.
  
  Resumidamente, temos este trabalho como um compilado básico contendo algumas informações referentes ao contexto histórico em que se encontrou a linguagem R em sua origem e seus usos na área da computação. Também são demonstradas algumas das diversas estruturas básicas e (quase) avançadas da linguagem R. Vejo que quanto aos capítulos de aplicações e ferramentas, talvez eles pudessem ter ordem inversa, sendo assim, a primeira aplicação que apresenta alguns conceitos, poderia ser mesclado aos capítulos de estruturas básicas e avançadas, assim dando uma visão um pouco mais direta de como as estruturas funcionam na prática.