\chapter{ Conceitos avançados da Linguagem R}
  \begin{comment}
    Prof. Dr. Ausberto S. Castro Vera
    UENF - CCT - LCMAT - Curso de Ciência da Computação
    Campos, RJ,  2021
    Disciplina: Paradigmas de Linguagens de Programação
    Data de entrega: 14 de out. 13:00
    C. Terceira Parte - Aspectos Avanzados
    módulos, funções, objetos, clases, subrotinas, etc.
    Segundo \cite{Sebesta2018}, a linguagem R
  \end{comment}
	Todos os conceitos avançados abordados abaixo foram retirados do \href{https://cran.r-project.org/doc/manuals/r-release/fullrefman.pdf}{manual disponibilizado pelo próprio time de criadores da linguagem} \cite{Team2021}.
  \section{Módulos}
    \subsection{Ambiente Global}
      O ambiente global é a raiz da área de trabalho do usuário. Um operação de atribuição a partir da linha de comando fará com que o objeto relevante pertença ao ambiente global. 
    \subsection{Ambiente léxico}
      Cada chamada para um função cria um quadro que contém as variáveis locais criadas na função e é avaliado em um ambiente, que em combinação cria um novo ambiente.
      
      Observe a terminologia: um quadro é um conjunto de variáveis, um ambiente é um aninhamento de quadros (ou de forma equivalente: o quadro mais interno mais o ambiente envolvente).
      
      Os ambientes podem ser atribuídos a variáveis ou estar contidos em outros objetos. No entanto, observe que eles não são objetos padrão - em particular, eles não são copiados na atribuição.
    
    \subsection{A pilha de chamadas}
      Cada vez que um função é chamada, um novo quadro de avaliação é criado. A qualquer momento durante a computação, os ambientes atualmente ativos podem ser acessados por meio da pilha de chamadas . Cada vez que uma função é chamada, uma construção especial chamada de contexto é criada internamente e colocada em uma lista de contextos. Quando uma função termina de avaliar, seu contexto é removido da pilha de chamadas.
      
      Tornar as variáveis definidas mais altas na pilha de chamadas disponível é chamado escopo dinâmico. A ligação para uma variável é então determinada pela definição mais recente (no tempo) da variável. Isso contradiz as regras de escopo padrão em R, que usam as ligações no ambiente no qual a função foi definida (escopo léxico). Algumas funções, especialmente aquelas que usam e manipulam fórmulas de modelo, precisam simular o escopo dinâmico acessando diretamente a pilha de chamadas.
      
  \section{Funções}
    Durante o uso de R para análise de dados, a maioria dos usuários se encontram querendo escrever suas próprias funções. Uma das vantagens do R é que eles podem até mudar o nome das funções em nível de sistema. Uma função pode ser criada seguindo a seguinte estrutura:
		
		\color{blue}
		\begin{verbatim}
    nomeDaFunção <- function(argumento1, argumento2, argumentoN) {expressões}

\end{verbatim}
\color{black}
    %aprofundar quanto ao argumento ... e ao escopos
  
  \section{Objetos}    
    \subsection{Objeto Vetor}
      Vetores podem ser entendidos como células adjacentes que contém dados. Essas células podem ser acessadas a partir de um índice, tal qual x[5]. R tem 6 tipos básicos de vetores atômicos: lógico, inteiro, real, complexo, string (ou caractere) e cru (raw).
    
    \subsection{Objetos da Linguagem}
      Existem três tipos de objetos que constituem a linguagem R. Elas são: chamadas (calls), expressões (expressions), e nomes (names). Expressões que são sintaticamente corretas são chamadas de afirmações (statements).
    
    \subsection{Objeto Expressão}
      Em R, pode-se ter objetos do tipo "expression". Uma expressão contém uma ou mais instruções. Uma declaração é uma coleção sintaticamente correta de tokens. Os objetos de expressão são objetos de linguagem especial que contêm instruções R analisadas, mas não avaliadas. A principal diferença é que um objeto de expressão pode conter várias dessas expressões. Outra diferença mais sutil é que os objetos do tipo "expression" são apenas avaliado quando explicitamente passado para eval, enquanto outros objetos de linguagem podem ser avaliados em alguns casos inesperados.
      
      Um objeto de expressão se comporta de maneira muito semelhante a uma lista e seus componentes devem ser acessados da mesma forma que os componentes de uma lista.
    
    \subsection{Objeto função}
      Em R, as funções são objetos e podem ser manipuladas da mesma maneira que qualquer outro objeto. Funções (ou mais precisamente, encerramentos de função) têm três componentes básicos: uma lista formal de argumentos, um corpo e um ambiente. A lista de argumentos é uma lista de argumentos separados por vírgulas. Um argumento pode ser usado para especificar um valor padrão para um argumento. Este valor será usado se a função for chamada sem nenhum valor especificado para aquele argumento. O ... argumento é especial e pode conter qualquer número de argumentos. Geralmente é usado se o número de argumentos for desconhecido ou nos casos em que os argumentos serão passados para outra função.
      
      O corpo é uma instrução R analisada. Geralmente é uma coleção de declarações entre colchetes, mas pode ser uma única declaração, um símbolo ou até mesmo uma constante.
      
      Uma função ambiente é o ambiente que estava ativo no momento em que a função foi criada. Todos os símbolos vinculados a esse ambiente são capturados e disponibilizados para a função. Essa combinação do código da função e das ligações em seu ambiente é chamada de 'encerramento de função', um termo da teoria da programação funcional. Neste documento, geralmente usamos o termo 'função', mas usamos 'encerramento' para enfatizar a importância do ambiente anexado.
      
      É possível extrair e manipular as três partes de um objeto de fechamento usando os construtos "formals", "body" e "environment" (todos os três podem também ser utilizadas no lado esquerdo das atribuições).
      
      Quando uma função é chamada, um novo ambiente (denominado ambiente de avaliação) é criado, cujo invólucro é o ambiente do fechamento da função. Esse novo ambiente é inicialmente preenchido com os argumentos não avaliados para a função; conforme a avaliação prossegue, as variáveis locais são criadas dentro dela.
      
      Também existe um recurso para converter funções de e para estruturas de lista usando "as.list" e "as.function". Eles foram incluídos para fornecer compatibilidade com a linguagem S e seu uso não é recomendado.
    
    \subsection{Objeto NULL (Nulo)}
      Existe um objeto especial chamado NULL. É utilizado sempre que houver necessidade de indicar ou especificar que um objeto está ausente. Não deve ser confundido com um vetor ou lista de comprimento zero.
      
      O objeto NULL não tem tipo nem propriedades modificáveis. Existe apenas um objeto NULL em R, ao qual todas as instâncias se referem. Para testar se algum valor é NULL, use "is.null". Você não pode definir atributos NULL.
    
    \subsection{Objetos Embutidos e Formas Especiais}
      Esses dois tipos de objeto contêm as funções da linguagem R, ou seja, aquelas que são exibidas como .Primitive em listagens de código. A diferença entre os dois está no tratamento do argumento. As funções embutidas têm todos os seus argumentos avaliados e passados para a função interna, de acordo com a chamada por valor , enquanto as funções especiais passam os argumentos não avaliados para a função interna.
      
      Da linguagem R, esses objetos são apenas outro tipo de função. A função "is.primitive" pode distinguí-los de funções interpretadas.
    
    \subsection{Objeto promessa (promise)}
      Objetos de promessa são parte do mecanismo de avaliação preguiçoso de R. Eles contêm três slots: um valor, uma expressão e um ambiente. Quando um função é chamada, os argumentos são correspondidos e, em seguida, cada um dos argumentos formais está vinculado a uma promessa. A expressão fornecida para aquele argumento formal e um ponteiro para o ambiente a partir do qual a função foi chamada são armazenados na promessa.
      
      Até que esse argumento seja acessado, não há nenhum valor associado à promessa. Quando o argumento é acessado, a expressão armazenada é avaliado no ambiente armazenado, e o resultado é retornado. O resultado também é salvo pela promessa. A função "substitute" irá extrair o conteúdo do slot de expressão. Isso permite que o programador acesse o valor ou a expressão associada à promessa.
      
      Dentro da linguagem R, os objetos de promessa são quase apenas vistos implicitamente: os argumentos reais das funções são desse tipo. Também existe uma função "delayedAssign" que fará uma promessa a partir de uma expressão. Geralmente, não há como no código R verificar se um objeto é uma promessa ou não, nem há uma maneira de usar o código R para determinar o ambiente de uma promessa.

    \subsection{Objeto ambiente}
      Os ambientes podem ser considerados como consistindo em duas coisas. Um quadro , que consiste em um conjunto de pares símbolo-valor de  e um invólucro , um ponteiro para um ambiente delimitador. Quando R procura o valor de um símbolo, o quadro é examinado e, se um símbolo correspondente for encontrado, seu valor será retornado. Caso contrário, o ambiente envolvente é então acessado e o processo repetido. Os ambientes formam uma estrutura em árvore na qual os gabinetes desempenham o papel de pais. A árvore de ambientes está enraizada em um vazio ambiente, disponível através de emptyenv(), que não tem pai. É o pai direto do ambiente do pacote básico (função disponível através da baseenv()).
      
      Os ambientes são criados implicitamente por chamadas de função, conforme descrito em Objetos de função e ambiente lexical . Nesse caso, o ambiente contém as variáveis locais para a função (incluindo os argumentos) e seu invólucro é o ambiente da função chamada atualmente. Ambientes também podem ser criados diretamente por new.env. O conteúdo do quadro de um ambiente pode ser acessada através da utilização de "ls", "names", "\$", "[", "[[", "get", e "get0", e manipulado por "\$<-", "[[<-" e associado bem como "eval" e  "evalq".
      
      A função "parent.env" pode ser usada para acessar um ambiente.
      
      Ao contrário da maioria dos outros objetos R, os ambientes não são copiados quando passados para funções ou usados em atribuições. Assim, se você atribuir o mesmo ambiente a vários símbolos e alterar um, os outros também mudarão. Em particular, atribuir atributos a um ambiente pode levar a surpresas.
    
    \subsection{Objeto Lista Pareada}
      Uma lista pareada (\textit{pairlist}) são usadas internamente na linguagem R, mas raramente visíveis no código interpretado. Pode ser criada pela função pairlist. Uma lista pareada de tamanho zero é nula. Mesmo sendo nula, cada objeto possui três espaços: o valor de CAR, o valor de CDR e o valor de TAG. O valor de TAG é uma linha de texto e CAR e CDR geralmente representam, respectivamente, o item no início da lista (cabeça) e o restante da lista (cauda) com um objeto NULL como valor final
      
      As lista pareadas são gerenciadas pela linguagem R da mesma forma que os vetores genéricos (listas). Os elementos são acessados usando a mesma sintaxe [[]]. O uso de listas pareadas está descontinuadas, visto que o uso de vetores genéricos são geralmente mais eficientes de se usar. Quando uma lista pareada interna é acessada do R, geralmente ela é convertida a um vetor genérico.
    
  \section{Classes}
  	Como dito em \cite{Team2021b} R tem um sistema de classes elaborado, controlado principalmente por meio do atributo de classe. Este atributo é um vetor de caracteres que contém a lista de classes das quais um objeto herda. Isso forma a base da funcionalidade de "métodos genéricos" em R.
    
    Este atributo pode ser acessado e manipulado virtualmente sem restrição pelos usuários. Não há verificação de que um objeto realmente contém os componentes que os métodos de classe esperam. Assim, a alteração do atributo de classe deve ser feita com cautela, e devem ser preferidas as funções específicas de criação e coerção quando estiverem disponíveis.
    
  \section{Sub-rotinas}
    Como também dito por \cite{Team2021b}, a sub-rotina (ou subprograma) define uma função usando uma sequência de declarações e definições que podem ser chamadas em diferentes localizações do programa. Quando duas tarefas executam a mesma sequência de operações, a sub-rotina associada a estes parâmetros é geralmente compartilhada pelas duas tarefas.
    
    R pertence a uma classe de linguagens de programação nas quais as sub-rotinas têm a capacidade de modificar ou construir outras sub-rotinas e avaliar o resultado como parte integrante da própria linguagem. Isso é semelhante ao Lisp e Scheme e outras linguagens da variedade de "programação funcional", mas em contraste com FORTRAN e a família ALGOL. A família Lisp leva este recurso ao extremo pelo paradigma "tudo é uma lista" no qual não há distinção entre programas e dados.
    
    R apresenta uma interface mais amigável para a programação do que o Lisp, pelo menos para alguém acostumado com fórmulas matemáticas e estruturas de controle do tipo C, mas o motor é realmente muito parecido com o Lisp. R permite acesso direto a expressões e funções e também permite que você as altere e posteriormente as execute, ou crie funções inteiramente novas a partir do zero.

    Há uma série de aplicações padrão dessa facilidade, como cálculo de derivadas analíticas de expressões ou a geração de funções polinomiais a partir de um vetor de coeficientes.
    No entanto, também existem usos que são muito mais fundamentais para o funcionamento da parte interpretada de R. Alguns deles são essenciais para a reutilização de funções como componentes em outras funções, como as chamadas (reconhecidamente não muito bonitas) ao "model.frame" que são construídas em várias rotinas de modelagem e plotagem. Outros usos simplesmente permitem interfaces elegantes para funcionalidade útil. Como exemplo, consideremos uma função de curva, que permite desenhar o gráfico de uma função dada como uma expressão como seno(x) ou as facilidades para plotar expressões matemáticas.
