\begin{comment}
  Prof. Dr. Ausberto S. Castro Vera
  UENF - CCT - LCMAT - Curso de Ciência da Computação
  Campos, RJ,  2021
  Disciplina: Paradigmas de Linguagens de Programação
\end{comment}
\newpage
\noindent
\textbf{Disciplina:}  \textit{Paradigmas de Linguagens de Programação 2021}\\
\textbf{Linguagem:}   \textit{Linguagem R}\\
\textbf{Aluno:}       \textit{João Vítor Fernandes Dias}\\
\textbf{Data:}        \today

\section*{Ficha de avaliação:}
  \begin{tabular}{|p{12cm}|c|}

    \hline
      % after \\: \hline or \cline{col1-col2} \cline{col3-col4} ...
      \textbf{Aspectos de avaliação (requisitos mínimos)}
      &
      \textbf{Pontos} \\
    \hline
      Elementos básicos da linguagem (Máximo: 01 pontos)
      & \\
      $\bullet$ Sintaxe (variáveis, constantes, comandos, operações, etc.)
      & \\
      $\bullet$ Usos e áreas de Aplicação da Linguagem
      & \\
    \hline
      Cada elemento da linguagem (definição) com exemplos (Máximo: 02 pontos)
      & \\
      $\bullet$ Exemplos com fonte diferenciada ( Courier , 10 pts, azul)
      & \\
    \hline
      Mínimo 5 exemplos completos - Aplicações (Máximo : 2 pontos)
      & \\
      $\bullet$ Uso de rotinas-funções-procedimentos, E/S formatadas
      & \\
      $\bullet$ Menu de operações, programas gráficos, matrizes, aplicações
      & \\
    \hline
      Ferramentas (compiladores, interpretadores, etc.) (Máximo : 2 pontos)
      & \\
      $\bullet$ Ferramentas utilizadas nos exemplos: pelo menos DUAS
      & \\
      $\bullet$ Descrição de Ferramentas existentes:  máximo 5
      & \\
      $\bullet$ Mostrar as telas dos exemplos junto ao compilador-interpretador
      & \\
      $\bullet$ Mostrar as telas dos resultados obtidos nas ferramentas
      & \\
      $\bullet$ Descrição das ferramentas (autor, versão, homepage, tipo, etc.)
      & \\
    \hline
      Organização do trabalho (Máximo: 01 ponto)
      & \\
      $\bullet$ Conteúdo, Historia, Seções, gráficos, exemplos, conclusões, bibliografia
      & \\
    \hline
      Uso de Bibliografia (Máximo: 01 ponto)
      & \\
      $\bullet$ Livros: pelo menos 3
      & \\
      $\bullet$ Artigos científicos: pelo menos 3 (IEEE Xplore, ACM Library)
      & \\
      $\bullet$ Todas as Referências dentro do texto, tipo [ABC 04]
      & \\
      $\bullet$ Evite Referências da Internet
      & \\
    \hline
      & \\
      Conceito do Professor (Opcional: 01 ponto)
      & \\
    \hline
      & \\
      \hfill Nota Final do trabalho:
      & \\
    \hline
  \end{tabular}\\
  \textit{Observação:} Requisitos mínimos significa a \textit{metade} dos pontos
